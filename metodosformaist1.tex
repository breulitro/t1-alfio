\documentclass{article}
\usepackage[brazil]{babel}
\usepackage[utf8]{inputenc}
\usepackage[T1]{fontenc}
\usepackage{sbc-template}
\usepackage{amsfonts}
\usepackage{amsmath}

\begin{document}
\begin{center}
Pontifícia Universidade Católica do Rio Grande do Sul\\
Cristiano Bolla Fernandes\\
Benito Michelon e Silva
\end{center}
\section{Introdução}

\section{Primeiro Problema}
O primeiro problema consiste em definir um algorítmo recursivo para obter a potenciação sobre o conjunto dos números naturais.
Foi requisitado dois algorítmos, um recursivo na cauda e o outro não.\\
Para ambos será exposta uma computação para fins de demonstrar a correteza do algorítmo.\\
pow: $\mathbb{N}$x$\mathbb{N}$$\rightarrow \mathbb{N}$\\
pow(b, 0) = 1\\
pow(b, e+1) = mult(b, pow(b, e))\\

Computação:\\
\begin{align*}
pow(2,3) &= mult(2, pow(2, 2))\\
&= mult(2, mult(2, pow(2, 1)))\\
&= mult(2, mult(2, mult(2, pow(2, 0))))\\
&= mult(2, mult(2, mult(2, 1)))\\
&= mult(2, mult(2, 2))\\
&= mult(2, 4)\\
&= 8
\end{align*}

\noindent powc: $\mathbb{N}$x$\mathbb{N}$$\rightarrow \mathbb{N}$\\
powc(b, e) = powacc(b, e, 1)\\
onde\\
\indent powacc: $\mathbb{N}$x$\mathbb{N}$x$\mathbb{N}$$\rightarrow \mathbb{N}$\\
\indent powacc(b, 0, a) = a\\
\indent powacc(b, e+1, a) = powacc(b, e, mult(b, a))\\

Computação:\\
\begin{align*}
powc(2, 3) &= powacc(2, 3, 1)\\
&= powacc(2, 2, 2)\\
&= powacc(2, 1, 4)\\
&= powacc(2, 0, 8)\\
&= 8
\end{align*}
\end{document}
