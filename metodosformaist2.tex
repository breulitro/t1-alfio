\documentclass{article}
\usepackage[brazil]{babel}
\usepackage[utf8]{inputenc}
\usepackage[T1]{fontenc}
\usepackage{sbc-template}
\usepackage{amsfonts}
\usepackage{amsmath}
\usepackage{amssymb}

\title{Trabalho 1 de Métodos Formais para a Computação}
\author{Cristiano Bolla Fernandes, Benito Michelon e Silva}
\address{Faculdade de Informática -- Pontifícia Universidade Católica do Rio Grande do Sul (PUCRS)}

\begin{document}
\maketitle

\section{Introdução}
Este trabalho tem por objetivo solidificar os conhecimentos adquiridos na disciplina de Métodos Formais para a Computação.
O trabalho será realizado com base no trabalho 1, escolhendo 2 dos 3 problemas resolvidos para provar por indução a sua correção parcial.
Os problemas escolhidos foram a potenciação e o somatório.

\section{Primeiro Problema}
O primeiro problema consiste em definir um algoritmo recursivo para obter a potenciação sobre o conjunto dos números naturais.
Foi requisitado dois algoritmos, um recursivo na cauda e o outro não.\\
Para ambos será exposta uma computação para fins de demonstrar a correção do algoritmo.

%%%%%%%%%%%%%%%%%%%%%%%%%%%%%%%%%%%%%%%%%%%%%%%%%%%%%%%%
% powc: Descrição
%%%%%%%%%%%%%%%%%%%%%%%%%%%%%%%%%%%%%%%%%%%%%%%%%%%%%%%%
\noindent \textbf{pow:} $\mathbb{N}$x$\mathbb{N}$$ \rightarrow \mathbb{N}$\\
\textbf{requer:} $\top$\\
\textbf{garante:} $\forall b:\mathbb{N}. \forall e:\mathbb{N}. pow(b, e) = b^e$\\
$pow(0, e) = 0$ \\
$pow(b, 0) = 1$ \\
$pow(b, e+1) = mult(b, pow(b, e))$ \\

%%%%%%%%%%%%%%%%%%%%%%%%%%%%%%%%%%%%%%%%%%%%%%%%%%%%%%%%
% powc: Prova do Caso Base + Caso Indutivo
%%%%%%%%%%%%%%%%%%%%%%%%%%%%%%%%%%%%%%%%%%%%%%%%%%%%%%%%
\noindent Iremos provar que $\forall b:\mathbb{N}.\:\forall e:\mathbb{N}.\:\top\rightarrow pow(b, e) = b^e$. \\
Pela tabela verdade da implicação, é suficiente mostrar que: \\
\begin{center}
$P \triangleq \forall b:\mathbb{N} \forall e:\mathbb{N}. pow(b, e) = b^e$\\
$P(x) \triangleq \forall b:\mathbb{N}. pow(b, x) = b^x$\\
\end{center}
\textbf{Prova do Caso Base:} Precisamos mostrar $P(0)$, isto é, $\forall b:\mathbb{N}. pow(b, 0) = b^0$\\
Seja $n \in \mathbb{N}$ uma variável arbitrária. É suficiente mostrar que $pow(n, 0) = n^0$.\\
Agora veja que:
\begin{align*}
LHS &= pow(n, 0) && (def. LHS)\\
&= 1 && (def. pow)\\
&= n^0 && (aritm.)\\
&= RHS\\
& q.e.d
\end{align*}

\noindent \textbf{Prova do Caso Indutivo:} Seja $x \in \mathbb{N}$ uma variável arbitrária. Então, assumimos como hipótese de indução (HI)
$\forall b:\mathbb{N}. pow(b, x) = b^x$.\\
Precisamos mostrar $P(x+1)$, isto é, $\forall b:\mathbb{N}. pow(b, x+1) = b^{x+1}$.\\
Seja $a \in \mathbb{N}$ uma variável arbitrária. Então é suficiente mostrar que $pow(a, x+1) = a^{x+1}$\\
Agora veja que:
\begin{align*}
RHS &= a^{x+1} && (def. RHS)\\
&= a^x * a && (algebra)\\
&= pow(a, x) * a && (HI)\\
&= mult(a, pow(a, x)) && (def. mult)\\
&= pow(a, x+1) && (def. pow)\\
&= LHS\\
& q.e.d
\end{align*}

%%%%%%%%%%%%%%%%%%%%%%%%%%%%%%%%%%%%%%%%%%%%%%%%%%%%%%%%
% powc: Descrição
%%%%%%%%%%%%%%%%%%%%%%%%%%%%%%%%%%%%%%%%%%%%%%%%%%%%%%%%
\noindent \textbf{powc:} $\mathbb{N}$x$\mathbb{N}$$ \rightarrow \mathbb{N}$\\
\textbf{requer:} $\top$\\
\textbf{garante:} $\forall b:\mathbb{N}. \forall e:\mathbb{N}. powc(b, e) = b^e$\\
$powc(b, e) = powacc(b, e, 1)\\$
\textbf{onde}\\
\indent \textbf{powacc:} $\mathbb{N}$x$\mathbb{N}$x$\mathbb{N}$$ \rightarrow \mathbb{N}$\\
\indent \textbf{requer:} $\top$\\
\indent \textbf{garante:} $\forall b:\mathbb{N}. \forall e:\mathbb{N}. powacc(b, e, 1) = b^e$\\
\indent \textbf{invariante:} $\forall k:\mathbb{N}.\: powacc(b_0,e_0,a_0) = b_k^{e_k}*a_k$\\
\indent $powacc(b, 0, a) = a\\
\indent powacc(b, e+1, a) = powacc(b, e, mult(b, a))\\$


%%%%%%%%%%%%%%%%%%%%%%%%%%%%%%%%%%%%%%%%%%%%%%%%%%%%%%%%
% powc: Caso base e caso indutivo
%%%%%%%%%%%%%%%%%%%%%%%%%%%%%%%%%%%%%%%%%%%%%%%%%%%%%%%%
\noindent Iremos provar que $\forall b:\mathbb{N}.\forall e:\mathbb{N}.\: \top \rightarrow powc(b, e) = b^e$. Pela tabela verdade da implicação é suficiente mostrar que:\\
\begin{center}
$P \triangleq \forall b:\mathbb{N}.\forall e:\mathbb{N}.\: powc(b, e) = b^e$\\
$P(x) \triangleq \forall b:\mathbb{N}.\: powc(b, x) = b^x$
\end{center}

\noindent\textbf{Prova do caso Base:} Precisamos mostrar $P(0)$, isto é,
$\forall b:\mathbb{N}.\: powc(b, 0) = b^0$\\
Seja n $\in \mathbb{N}$ uma variável arbitrária. É suficiente mostrar que $powc(n, 0) = n^0$.\\
Agora veja que:
\begin{align*}
LHS &= powc(n, 0) && (def. LHS)\\
&= powacc(n, 0, 1) && (def. powc)\\
&= 1 && (def. powacc)\\
&= n^0 && (aritmetica)\\
&= RHS\\
& q.e.d
\end{align*}

\noindent\textbf{Prova do caso Indutivo:}
Seja $x \in \mathbb{N}$ uma variável arbitrária. Então assumimos como hipótese de indução (HI)
$\forall b:\mathbb{N}. powc(b, x) = b^x$.\\
Precisamos mostrar $P(x+1)$, isto é, $\forall b:\mathbb{N}. powc(b, x+1) = b^{x+1}i$.\\
Seja n $\in \mathbb{N}$ uma variável arbitrária. Então é suficiente mostrar que $powc(n, x+1) = n^{x+1}$\\
Agora veja que:
\begin{align*}
LHS &= powc(n, x+1) && (def. LHS)\\
&= powacc(n, x+1, 1) && (def. powc)\\
&= powacc(n, x, mult(n, 1)) && (def. powacc)\\
&= powacc(n, x, n) && (def. powacc)\\
&= n^x * n && (spec. powacc)\\
&= n^{x+1} && (algebra)\\
&= RHS\\
& q.e.d
\end{align*}


%%%%%%%%%%%%%%%%%%%%%%%%%%%%%%%%%%%%%%%%%%%%%%%%%%%%%%%%
% powacc: Caso base e caso indutivo
%%%%%%%%%%%%%%%%%%%%%%%%%%%%%%%%%%%%%%%%%%%%%%%%%%%%%%%%
\noindent Iremos demonstrar que
$\forall b:\mathbb{N}. \forall e:\mathbb{N}.\: \top \rightarrow powacc(b, e, r) = b^e * r$.
Pela tabela verdade da implicação é suficiente mostrar que:\\
\begin{center}
$P \triangleq \forall b:\mathbb{N}. \forall e:\mathbb{N}.\: powacc(b, e, r) = b^e * r$\\
$P(x) \triangleq \forall b:\mathbb{N}.\: powacc(b, x, r) = b^x * r$\\
\end{center}

\noindent\textbf{Prova do caso Base:}
Precisamos mostrar $P(0)$, isto é, $\forall b:\mathbb{N}.\: powacc(b, 0, r) = b^0 * r$\\
Seja $n \in \mathbb{N}$ uma variável arbitrária. É suficiente mostrar que $powacc(n, 0, r) = n^0 * r$.\\
Agora veja que:
\begin{align*}
RHS &= n^0 * r && (def. RHS)\\
&= 1 * r && (aritmetica)\\
&= r && (aritmetica)\\
&= powacc(n, 0, r) && (def. powacc)\\
&= LHS\\
& q.e.d
\end{align*}

\noindent\textbf{Prova do caso indutivo:}
Seja $x \in \mathbb{N}$ uma variável arbitrária. Então, assumimos como hipótese de indução (HI)
$\forall b:\mathbb{N}. powacc(b, x, r) = b^x * r$.
Precisamos mostrar que $P(x+1)$, isto é, $\forall b: \mathbb{N}.\: powacc(b, x+1, r) = b^{x+1} * r$.
Seja n $\in \mathbb{N}$ uma variável arbitrária. Então é suficiente mostrar que $powacc(n, x+1, r) = n^{x+1} * r$\\
Agora veja que:
\begin{align*}
LHS &= powacc(n, x+1, r) && (def. LHS)\\
&= powacc(n, x, n * r) && (def. powacc)\\
&= n^x * n * r && (HI)\\
&= n^{x+1} * r && (algebra)\\
&= RHS\\
& q.e.d
\end{align*}

%%%%%%%%%%%%%%%%%%%%%%%%%%%%%%%%%%%%%%%%%%%%%%%%%%%%%%%%
% powacc: Prova da Invariante (Base + indução)
%%%%%%%%%%%%%%%%%%%%%%%%%%%%%%%%%%%%%%%%%%%%%%%%%%%%%%%%
\noindent Queremos demonstrar que
$\forall k:\mathbb{N}.\: \top \rightarrow powacc(b_0, e_0, a_0) = b_k^{e_k} * a_k$.
Pela tabela verdade da implicação é suficiente mostrar que:
\begin{center}
$P \triangleq \forall k:\mathbb{N}.\: powacc(b_0, e_0, a_0) = b_k^{e_k} * a_k$ \\
$P(j) \triangleq powacc(b_0, e_0, a_0) = b_j^{e_j} * a_j$ \\
\end{center}
Precisamos mostrar $P(0)$, isto é, $powacc(b_0, e_0, a_0) = b_0^{e_0} * a_0$.
Como $P(0)$ representa uma execução com zero chamadas recursivas. Logo, por definição de powacc, $e_0 = 0$ \\
Agora veja que:
\begin{align*}
LHS &= powacc(b_0, 0, a_0) && (def. LHS)\\
&= a_0 && (def. powacc)\\
&= b_0^0 * a_0 && (aritmetica)\\
&= RHS\\
& q.e.d
\end{align*}


\section{Segundo Problema}
O segundo problema é fazer um algoritmo que compute um somatório.
Foi requisitado dois algoritmos, um recursivo na cauda e o outro não. \\
Para ambos será exposta uma computação para fins de demonstrar a correção do algoritmo.

%%%%%%%%%%%%%%%%%%%%%%%%%%%%%%%%%%%%%%%%%%%%%%%%%%%%%%%%
% somai: Descrição
%%%%%%%%%%%%%%%%%%%%%%%%%%%%%%%%%%%%%%%%%%%%%%%%%%%%%%%%
\noindent \textbf{somai:} $\mathbb{Z}$x$\mathbb{Z}$$ \rightarrow \mathbb{Z}$\\
\textbf{requer:} $\top$ \\
\textbf{garante:} $\forall x:\mathbb{Z}. \forall y:\mathbb{Z}. somai(x_0,y_0) = \sum\limits_{x_0 \le i \le y_0} i$\\

%%%%%%%%%%%%%%%%%%%%%%%%%%%%%%%%%%%%%%%%%%%%%%%%%%%%%%%%
% somai: Prova do Caso Base + Caso Indutivo
%%%%%%%%%%%%%%%%%%%%%%%%%%%%%%%%%%%%%%%%%%%%%%%%%%%%%%%%
\noindent Iremos provar que $\forall b:\mathbb{N}.\:\forall e:\mathbb{N}.\:\top\rightarrow pow(b, e) = b^e$.
Pela tabela verdade da implicação, é suficiente mostrar que:

\begin{center}
$P \triangleq \forall b:\mathbb{N} \forall e:\mathbb{N}. pow(b, e) = b^e$ \\
$P(x) \triangleq \forall b:\mathbb{N} pow(b, x) = b^x$ \\
\end{center}

\noindent \textbf{Prova do Caso Base:} Precisamos mostrar P(0), isto é, $\forall b:\mathbb{N} pow(b, 0) = b^0$ \\
Seja n $\in$ $\mathbb{N}$ uma variável arbitrária. É suficiente mostrar que $pow(n, 0) = n^0$. \\
Agora veja que:
\begin{align*}
LHS &= \sum\limits_{x_0 \le i \le x_0} i && (def. LHS) \\
&= x_0 && (def. \sum) \\
&= somai(x_0, x_0) && (def. somai) \\
&= RHS \\
& q.e.d
\end{align*}

\noindent \textbf{Prova do Caso Indutivo:} Seja $x \in \mathbb{N}$ uma variável arbitrária. Então, assumimos como hipótese de indução (HI)
$\forall b:\mathbb{N}. pow(b, x) = b^x$. \\
Precisamos mostrar $P(x+1)$, isto é, $\forall b:\mathbb{N}. pow(b, x+1) = b^{x+1}$. \\
Seja a $\in \mathbb{N}$ uma variável arbitrária. Então é suficiente mostrar que $pow(a, x+1) = a^{x+1}$ \\
Agora veja que:
\begin{align*}
RHS &= a^{x+1} && (def. RHS)\\
&= a^x * a && (algebra)\\
&= pow(a, x) * a && (HI)\\
&= mult(a, pow(a, x)) && (def. mult)\\
&= pow(a, x+1) && (def. pow)\\
&= LHS\\
& q.e.d
\end{align*}

%%%%%%%%%%%%%%%%%%%%%%%%%%%%%%%%%%%%%%%%%%%%%%%%%%%%%%%%
% somaic: Descricao
%%%%%%%%%%%%%%%%%%%%%%%%%%%%%%%%%%%%%%%%%%%%%%%%%%%%%%%%
\noindent \textbf{somaic:} $\mathbb{Z}$x$\mathbb{Z}$$ \rightarrow \mathbb{Z}$\\
\textbf{requer:} i $\leq$ s\\
\textbf{garante:} somaic(i, s) = $\sum\limits_{i \le k \le s} k$\\
$somaic(i, s) = somacc(i, s, 0)\\$
\\
onde\\
\indent \textbf{somacc:} $\mathbb{Z}$x$\mathbb{Z}$x$\mathbb{Z}$$ \rightarrow \mathbb{Z}$\\
\indent \textbf{requer:} i $\leq$ s \& a = 0\\
\indent \textbf{garante:} somaic(i, s, a) = a + $\sum\limits_{i \le k \le s} k$\\
\indent \textbf{invariante:} $\forall k:\mathbb{N}.\: somacc(i_0, s_0, a_0) = somacc(i_k, s_k, a_k)$\\
\indent $somacc(i, i, a) = i + a\\
\indent somacc(i, s, a) = somacc(i, s-1, a+s)\\$
\\
%%%%%%%%%%%%%%%%%%%%%%%%%%%%%%%%%%%%%%%%%%%%%%%%%%%%%%%%
% somaic: Prova caso base + indutivo
%%%%%%%%%%%%%%%%%%%%%%%%%%%%%%%%%%%%%%%%%%%%%%%%%%%%%%%%
Iremos demonstrar que
$\forall i:\mathbb{N}. \forall s:\mathbb{N}. i \leq s\rightarrow somaic(i, s) = \sum\limits_{i \le k \le s} k$.
Assumindo que a pré condição seja satisfeita, logo temos que
$\forall i:\mathbb{N}. \forall s:\mathbb{N}.\: \top\rightarrow somaic(i, s) = \sum\limits_{i \le k \le s} k$,
Que pela tabela verdade da implicação vemos que é suficiente mostrar que:\\
\begin{center}
$P \triangleq \forall i:\mathbb{N}. \forall s:\mathbb{N}. somaic(i, s) = \sum\limits_{i \le k \le s} k$\\
$P(x) \triangleq \forall i:\mathbb{N}. somaic(i, x) = \sum\limits_{i \le k \le x} k$\\
\end{center}
\textbf{Prova do caso Base:} Precisamos demonstrar P(x) para x = i,
ou seja somaic(x, x) = $\sum\limits_{x \le k \le x} k$.\\
Agora veja que:
\begin{align*}
RHS &= \sum\limits_{i \le k \le s} k && (def. RHS)\\
&= x && (def. \sum)\\
&= somaic(x, x) && (def. somaic)\\
&= LHS \\
& q.e.d
\end{align*}
\textbf{Prova do caso Indutivo:} Seja x $\in \mathbb{N}$ uma variável arbitrária, então assumimos como hipótese
de indução (HI) $\forall i:\mathbb{N}. somaic(i, x) = \sum\limits_{i \le k \le x} k$. Precisamos mostrar P(x+1),
isto é, $\forall i:\mathbb{N}. somaic(i, x+1) = \sum\limits_{i \le k \le x+1} k$.
Seja $l \in \mathbb{N}$ uma variável arbitrária. É suficiente mostrar que
$somaic(l, x+1) = \sum\limits_{l \le k \le x+1} k$.\\
Agora veja que:\\
\begin{align*}
RHS &= \sum\limits_{l \le k \le x+1} k && (def. RHS)\\
&= \sum\limits_{l \le k \le x} k + (x +1) && (def. \sum) \\
&= somacc(l, x, x+1) && (spec. somacc)\\
&= somacc(l, x+1, 0) && (def. somacc)\\
&= somaic(l, x+1)	&& (def. somaic)\\
&= LHS\\
& q.e.d
\end{align*}
%%%%%%%%%%%%%%%%%%%%%%%%%%%%%%%%%%%%%%%%%%%%%%%%%%%%%%%%
% somacc: Prova caso base + indutivo
%%%%%%%%%%%%%%%%%%%%%%%%%%%%%%%%%%%%%%%%%%%%%%%%%%%%%%%%
Iremos demonstrar que
$\forall i:\mathbb{N}. \forall s:\mathbb{N}. \forall a:\mathbb{N}.\:  i \leq s \& a = 0 \rightarrow somacc(i, s, a) = a + \sum\limits_{i \le k \le s} k$.
Assumindo que a pré condição seja satisfeita, logo temos que
$\forall i:\mathbb{n}. \forall s:\mathbb{N}. \forall a:\mathbb{N}. \top\rightarrow somacc(i, s, a) = a + \sum\limits_{i \le k \le s} k$.
Que pela tabela verdade da implicação, vemos que é suficiente mostrar que:
\begin{center}
$P \triangleq \forall i:\mathbb{n}. \forall s:\mathbb{N}. \forall a:\mathbb{N}. somacc(i, s) = a + \sum\limits_{i \le k \le s} k$
$P(x) \triangleq \forall i:\mathbb{n}. \forall a:\mathbb{N}. somacc(i, x, a) = a + \sum\limits_{i \le k \le x} k$
\end{center}
\textbf{Prova do caso Base:} Precisamos demonstrar P(x) para x = i, ou seja somacc(x, x, a) = a + $\sum\limits_{x \le k \le x} k$.\\
Agora veja que:
\begin{align*}
RHS = a + \sum\limits_{x \le k \le x} k && (def. RHS)\\
&= a + x && (def. \sum)\\
&= sumacc(x, x, a) && (def. somacc)\\
&= LHS\\
& q.e.d
\end{align*}
\textbf{Prova do caso Indutivo:} Seja $x \in \mathbb{N}$ uma variável arbitrária, então assumimos como hipótese
de indução (HI) $\forall i:\mathbb{N}.\forall a:\mathbb{N} somacc(i, x, a) = a + \sum\limits_{i \le k \le x} k.$
Precisamos mostrar P(x+1), isto é,
$\forall i:\mathbb{N}.\forall a:\mathbb{N} somacc(i, x+1, a) = a + \sum\limits_{i \le k \le x+1} k.$
Seja $l \in \mathbb{N}$ uma variável arbitrária. Precisamos  mostrar que
$\forall \forall a:\mathbb{N} somacc(l, x+1, a) = a + \sum\limits_{l \le k \le x+1} k.$
Tendo também $r \in \mathbb{N}$ uma variável arbitrária, temos que é suficiente mostrar
$somacc(l, x+1, r) = r + \sum\limits_{l \le k \le x+1} k.$\\
Agora veja que:
\begin{align*}
LHS &= somacc(l, x+1, r) && (def. LHS)\\
&= somacc(l, x, a+(x+1)) && (def. somacc)\\
&= \sum\limits_{l \le k \le x} k + (x+1) + a && (HI)\\
&= \sum\limits_{l \le k \le x+1} k + a && (def. \sum)\\
&= RHS\\
& q.e.d
\end{align*}
\section{Conclusão}
Conforme foi dito na introdução, resolvemos os três problemas com duas soluções cada, uma recursiva na cauda e outra não.
Demonstramos o funcionamento de cada uma das soluções mostrando a execução dos algoritmos e chegamos nos resultados esperados.
As invariantes desempenham um papel crítico na evolução do desenvolvimento de software. A partir do conhecimento
adquirido em aula e através deste trabalho, percebemos que a identificação da invariante pode proteger um programador
de implementar algoritmos errados pela falta de compreensão do problema, além de fazer alterações que podem modificar o
comportamento correto de um programa, ou seja, não entender as invariantes que ocorrem
no problema torna muito fácil introduzir novos erros em um código já funcional.

\begin{thebibliography}{9}
\bibitem{invariants}
	Michael D. Ernst, Jake Cockrell, William G. Griswold, David Notkin.\\
	\emph{"Dynamically Discovering Likely Program Invariants to Support Program Evolution."}
	International Conference on Software Engineering, pp. 213–224. 1999.

\bibitem{logicincc}
	Michael Huth, Mark Ryan.\\
	\emph{"Logic in Computer Science.", Second Edition}.
\end{thebibliography}
\end{document}
