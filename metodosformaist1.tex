\documentclass{article}
\usepackage[brazil]{babel}
\usepackage[utf8]{inputenc}
\usepackage[T1]{fontenc}
\usepackage{sbc-template}
\usepackage{amsfonts}
\usepackage{amsmath}

\title{Trabalho 1 de Métodos Formais para a Computação}
\author{Cristiano Bolla Fernandes, Benito Michelon e Silva}
\address{Faculdade de Informática -- Pontifícia Universidade Católica do Rio Grande do Sul (PUCRS)}

\begin{document}
\maketitle

\section{Introdução}
Este trabalho tem por objetivo solidificar os conhecimentos adquiridos na disciplina de Métodos Formais para a Computação.
O trabalho consiste na criação de algoritmos para a resolução de três problemas. O primeiro é implementar um algoritmo que calcule a
potência de um número natural com um expoente natural também. O segundo problema é fazer um programa que realize o somatório e o
terceiro problema é implementar um algoritmo que compute o enézimo número da sequencia de Fibonacci.

\section{Primeiro Problema}
O primeiro problema consiste em definir um algoritmo recursivo para obter a potenciação sobre o conjunto dos números naturais.
Foi requisitado dois algoritmos, um recursivo na cauda e o outro não.\\
Para ambos será exposta uma computação para fins de demonstrar a correção do algoritmo.\\
\\
\textbf{pow:} $\mathbb{N}$x$\mathbb{N}$$ \rightarrow \mathbb{N}$\\
$pow(b, 0) = 1\\
pow(b, e+1) = mult(b, pow(b, e))\\$
\\

Computação:\\
\begin{align*}
pow(2,3) &= mult(2, pow(2, 2))\\
&= mult(2, mult(2, pow(2, 1)))\\
&= mult(2, mult(2, mult(2, pow(2, 0))))\\
&= mult(2, mult(2, mult(2, 1)))\\
&= mult(2, mult(2, 2))\\
&= mult(2, 4)\\
&= 8
\end{align*}

\noindent \textbf{powc:} $\mathbb{N}$x$\mathbb{N}$$ \rightarrow \mathbb{N}$\\
$powc(b, e) = powacc(b, e, 1)\\$
\\
onde\\
\indent \textbf{powacc:} $\mathbb{N}$x$\mathbb{N}$x$\mathbb{N}$$ \rightarrow \mathbb{N}$\\
\indent \textbf{invariante:} $\forall$i:$\mathbb{N}.\: powacc(b_0,e_0, a_0) = powacc(b_i, e_i, a_i)$\\
\indent $powacc(b, 0, a) = a\\
\indent powacc(b, e+1, a) = powacc(b, e, mult(b, a))\\$

Computação:\\
\begin{align*}
powc(2, 3) &= powacc(2, 3, 1)\\
&= powacc(2, 2, 2)\\
&= powacc(2, 1, 4)\\
&= powacc(2, 0, 8)\\
&= 8
\end{align*}

\section{Segundo Problema}
O segundo problema é fazer um algoritmo que compute um somatório.
Foi requisitado dois algoritmos, um recursivo na cauda e o outro não.\\
Para ambos será exposta uma computação para fins de demonstrar a correção do algoritmo.\\
\\
\noindent \textbf{somai:} $\mathbb{Z}$x$\mathbb{Z}$$ \rightarrow \mathbb{Z}$\\
$somai(i, i) = i\\
somai(i, s) = s + somai(i, s-1)\\$
\\
Computação:
\begin{align*}
somai(-3, -1) &= (-1) + somai(-3, -2)\\
&= (-1) + (-2) + somai(-3, -3)\\
&= (-1) + (-2) + (-3)\\
&= -6
\end{align*}
\noindent \textbf{somaic:} $\mathbb{Z}$x$\mathbb{Z}$$ \rightarrow \mathbb{Z}$\\
$somaic(i, s) = somacc(i, s, 0)\\$
\\
onde\\
\indent \textbf{somacc:} $\mathbb{Z}$x$\mathbb{Z}$x$\mathbb{Z}$$ \rightarrow \mathbb{Z}$\\
\indent \textbf{invariante:} $\forall k:\mathbb{N}.\: somacc(i_0, s_0, a_0) = somacc(i_k, s_k, a_k)$\\
\indent $somacc(i, i, a) = i + a\\
\indent somacc(i, s, a) = somacc(i, s-1, a+s)\\$
\\
Computação:
\begin{align*}
somaic(1, 5) &= somacc(1, 5, 0)\\
&= somacc(1, 4, 5)\\
&= somacc(1, 3, 9)\\
&= somacc(1, 2, 12)\\
&= somacc(1, 1, 14)\\
&= 15
\end{align*}

\section{Terceiro Problema}
O terceiro problema é calcular o enézimo número da série de Fibonacci. O próximo número da série é obtido pela soma do atual com o anterior.\\
Foi requisitado dois algoritmos, um recursivo na cauda e o outro não.\\
Para ambos será exposta uma computação para fins de demonstrar a correção do algoritmo.\\
\\
\textbf{fib:}$\mathbb{N}^* \rightarrow \mathbb{N}^*$\\
$fib(1) = 1\\
fib(2) = 1\\
fib(x+2) = fib(x+1) + fib(x)\\$
\\
Computação:
\begin{align*}
fib(5) &= fib(4) + fib(3)\\
&= fib(3) + fib(2) + fib(2) + fib(1)\\
&= fib(2) + fib(1) + 1 + 1 + 1\\
&= 1 + 1 + 1 + 1 + 1\\
&= 5
\end{align*}

\noindent\textbf{fibc:} $\mathbb{N}^* \rightarrow \mathbb{N}^*$\\
$fibc(n) = fibacc(n, 0, 1)\\$
\\
onde\\
\indent \textbf{fibacc:} $\mathbb{N}^*$x$\mathbb{N}$x$\mathbb{N}^* \rightarrow \mathbb{N}^*$\\
\indent \textbf{invariante:} $\forall k:\mathbb{N}.\: fibacc(n_0, p_0, a_0) = fibacc(n_k, p_k, a_k)$\\
\indent $fibacc(1, p, a) = a\\
\indent fibacc(n+1, p, a) = fibacc(n, a, p+a)\\$
\\
Computação:
x$_{m+1}$
\begin{align*}
fibc(6) &= fibacc(6, 0, 1)\\
&= fibacc(5, 1, 1)\\
&= fibacc(4, 1, 2)\\
&= fibacc(3, 2, 3)\\
&= fibacc(2, 3, 5)\\
&= fibacc(1, 5, 8) && (foo)\\
&= 8
\end{align*}

\section{Conclusão}
Conforme foi dito na introdução, resolvemos os três problemas com duas soluções cada, uma recursiva na cauda e outra não.
Demonstramos o funcionamento de cada uma das soluções mostrando a execução dos algoritmos e chegamos nos resultados esperados.
As invariantes desempenham um papel crítico na evolução do desenvolvimento de software. A partir do conhecimento
adquirido em aula e através deste trabalho, percebemos que a identificação da invariante pode proteger um programador
de implementar algoritmos errados pela falta de compreensão do problema, além de fazer alterações que podem modificar o
comportamento correto de um programa, ou seja, não entender as invariantes que ocorrem
no problema torna muito fácil introduzir novos erros em um código já funcional.

\begin{thebibliography}{9}
\bibitem{invariants}
	Michael D. Ernst, Jake Cockrell, William G. Griswold, David Notkin.\\
	\emph{"Dynamically Discovering Likely Program Invariants to Support Program Evolution."}
	International Conference on Software Engineering, pp. 213–224. 1999.

\bibitem{logicincc}
	Michael Huth, Mark Ryan.\\
	\emph{"Logic in Computer Science.", Second Edition}.
\end{thebibliography}
\end{document}
