\documentclass{article}
\usepackage[brazil]{babel}
\usepackage[utf8]{inputenc}
\usepackage[T1]{fontenc}
\usepackage{sbc-template}
\usepackage{amsfonts}
\usepackage{amsmath}

\title{Trabalho 1 de Métodos Formais para a Computação}
\author{Cristiano Bolla Fernandes, Benito Michelon e Silva}
\address{Faculdade de Informática -- Pontifícia Universidade Católica do Rio Grande do Sul (PUCRS)}

\begin{document}
\maketitle
\begin{center}
Pontifícia Universidade Católica do Rio Grande do Sul\\
Cristiano Bolla Fernandes\\
Benito Michelon e Silva
\end{center}
\section{Introdução}

\section{Primeiro Problema}
O primeiro problema consiste em definir um algorítmo recursivo para obter a potenciação sobre o conjunto dos números naturais.
Foi requisitado dois algorítmos, um recursivo na cauda e o outro não.\\
Para ambos será exposta uma computação para fins de demonstrar a correteza do algorítmo.\\
\textbf{pow:} $\mathbb{N}$x$\mathbb{N}$$ \rightarrow \mathbb{N}$\\
pow(b, 0) = 1\\
pow(b, e+1) = mult(b, pow(b, e))\\

Computação:\\
\begin{align*}
pow(2,3) &= mult(2, pow(2, 2))\\
&= mult(2, mult(2, pow(2, 1)))\\
&= mult(2, mult(2, mult(2, pow(2, 0))))\\
&= mult(2, mult(2, mult(2, 1)))\\
&= mult(2, mult(2, 2))\\
&= mult(2, 4)\\
&= 8
\end{align*}

\noindent \textbf{powc:} $\mathbb{N}$x$\mathbb{N}$$ \rightarrow \mathbb{N}$\\
powc(b, e) = powacc(b, e, 1)\\
onde\\
\indent \textbf{powacc:} $\mathbb{N}$x$\mathbb{N}$x$\mathbb{N}$$ \rightarrow \mathbb{N}$\\
\indent \textbf{invariante:} $\forall$i:$\mathbb{N}.\: powacc(b_0,e_0, a_0) = powacc(b_i, e_i, a_i)$\\
\indent powacc(b, 0, a) = a\\
\indent powacc(b, e+1, a) = powacc(b, e, mult(b, a))\\

Computação:\\
\begin{align*}
powc(2, 3) &= powacc(2, 3, 1)\\
&= powacc(2, 2, 2)\\
&= powacc(2, 1, 4)\\
&= powacc(2, 0, 8)\\
&= 8
\end{align*}

\section{Segundo Problema}
O segundo problema é fazer um algorítmo que compute um somatório.

\noindent \textbf{somai:} $\mathbb{Z}$x$\mathbb{Z}$$ \rightarrow \mathbb{Z}$\\
somai(i, i) = i\\
somai(i, s) = s + somai(i, s-1)\\
Computação:
\begin{align*}
somai(-3, -1) &= (-1) + somai(-3, -2)\\
&= (-1) + (-2) + somai(-3, -3)\\
&= (-1) + (-2) + (-3)\\
&= -6
\end{align*}
\noindent \textbf{somaic:} $\mathbb{Z}$x$\mathbb{Z}$$ \rightarrow \mathbb{Z}$\\
somaic(i, s) = somacc(i, s, 0)\\
onde\\
\indent \textbf{somacc:} $\mathbb{Z}$x$\mathbb{Z}$x$\mathbb{Z}$$ \rightarrow \mathbb{Z}$\\
\indent \textbf{invariante:} $\forall k:\mathbb{N}.\: somacc(i_0, s_0, a_0) = somacc(i_k, s_k, a_k)$\\
\indent somacc(i, i, a) = i + a\\
\indent somacc(i, s, a) = somacc(i, s-1, a+s)\\
\begin{align*}
somaic(1, 5) &= somacc(1, 5, 0)\\
&= somacc(1, 4, 5)\\
&= somacc(1, 3, 9)\\
&= somacc(1, 2, 12)\\
&= somacc(1, 1, 14)\\
&= 15
\end{align*}
\section{Terceiro Problema}
\section{Conclusão}
\end{document}
